Let us consider the following problem:

\quad \emph{Given a binary tree $t$ with integer values in the leaves.}

\quad \emph{Replace every value in $t$ by the minimum.}
\end{displaycode}

The most obvious way to solve this, would be by first traversing the tree to calculate the minimum and then traversing the tree to replace all values by that.
Note that it is simple to prove the termination and we go twice through the tree.

However, there is a more efficient solution to this problem \cite{bird1984}.
By using a \emph{cyclic} program, he described a program for which only one traversal is needed.
Normally, one defines functions on algebraic data types by using structural recursion and then proof assistants, such as Coq and Agda \cite{barras1997,norell2008y}, can automatically check the termination.
Cyclic programs, on the other hand, are \emph{not} structurally recursive and they do \emph{not} necessarily terminate.
One can show this particular one terminates using clocked type theory \cite{atkey2013}, but this has not been implemented in a proof assistant yet.
Hence, this solution is more efficient, but the price we pay, is that proving termination becomes more difficult.

In addition, showing correctness requires different techniques.
For structurally recursive functions, one can use structural induction.
For cyclic programs, that technique is not available and thus broader techniques are needed.

This pearl describes an Agda implementation of this program together with a proof that it is terminating and a correctness proof \cite{norell2008y}.
Our solution is based on the work by Atkey and McBride \cite{atkey2013} and the approach shows similarities to clocked type theory \cite{bahr2017clocks}.
We start by giving an Haskell implementation to demonstrate the issues we have to tackle.
After that we discuss the main tool for checking termination in Agda, namely \emph{sized types}.
Types are assigned sizes and if those decrease in recursive calls, then the program is productive.
We then give the solution, which is terminating since Agda accepts it.
Lastly, we demonstrate how to do proofs with sized types and we finish by proving correctness via equational reasoning. 